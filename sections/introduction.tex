\section{Introduction}
	
	\begin{frame}{Facts about hhuOS}
	\begin{itemize}
		\setlength\itemsep{1em}
		\item A small operating system for teaching and learning purposes
		\item written for x86 32-bit architecture
		\item written in C++ and x86-Assembler using gcc and nasm
		\item Open-Source, published under the GPL v3 license (\url{https://github.com/hhuOS/hhuOS})
		\item $\sim60000$ lines of code
	\end{itemize}
	\end{frame}
	
	\begin{frame}{hhuOS - Features}
		\begin{itemize}
			\item Processes \& Threads
			\begin{itemize}
				\item Round-Robin based preemptive scheduling
				\item Binary files are executed as processes
				\item Each process has its own address space
			\end{itemize}
			\vspace{1.0em}
			\item Adress spaces and memory management
			\begin{itemize}
				\item Using x86-Paging mechanism (virtual/physical memory)
				\item Lazy mapping implementation
				\item Kernel address space is always mapped below 128 MiB (not accessible from user space)
				\item Different memory managing algorithms are implemented (Free List, Bitmap, Table)
			\end{itemize}
		\end{itemize}
	\end{frame}

	\begin{frame}{hhuOS - Features}
		\begin{itemize}
			\item Unified Library
			\begin{itemize}
				\item Single codebase for user- and kernel-space library
				\item Functions requiring kernel access are outsourced into an interface (implemented two times - for kernel and user space)
				\item Kernel access from user space via system calls (software interrupts)
			\end{itemize}
			\vspace{1.0em}
			\item Filesystem
			\begin{itemize}
				\item Virtual Filesystem (can mount physical filesystems)
				\item FAT (using \href{http://elm-chan.org/fsw/ff/00index\_e.html}{FatFs})
				\item ISO9660 (bachelor thesis by Moritz Riefer)
			\end{itemize}
		\end{itemize}
	\end{frame}
	
	\begin{frame}{hhuOS - Features}
		\begin{itemize}
			\item Hardware Support:
			\begin{itemize}
				\item Basic I/O: Keyboard \& Mouse, Serial \& Parallel Ports, VESA \& CGA graphics, PCI \& ISA bus
				\item Interrupt: Programmable Interrupt Controller (PIC), APIC (bachelor thesis by Christoph Urlacher)
				\item Time: Programmable Interval Timer, Real Time Clock, APIC Timer, ACPI Timer
				\item Storage: Floppy, IDE (bachelor thesis by Tim Laursichkat; enhanced with ATAPI support by Moritz Riefer), AHCI (bachelor thesis by Manuel Angelescu)
				\item Sound: PC Speaker, SoundBlaster
				\item Network: Realtek RTL8139 (bachelor thesis by Alexander Hansen), Ne2000 (bachelor thesis by Marcel Thiel)
			\end{itemize}
			\item Further features:
			\begin{itemize}
				\item Multiboot2 compatible: Bootable on BIOS and UEFI systems
				\item Own UEFI bootloader (\href{https://github.com/hhuOS/towboot}{towboot}, developed by Niklas Sombert)
				\item UDP/IP-stack (based on bachelor thesis by Hannes Feil)
			\end{itemize}
		\end{itemize}
	\end{frame}