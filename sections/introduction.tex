\section{Introduction}
	
	\begin{frame}{Facts about hhuOS}
	\begin{itemize}
		\setlength\itemsep{1em}
		\item A small operating system for teaching and learning purposes
		\item written for x86 32-bit architecture
		\item written in C++ and x86-Assembler using gcc and nasm
		\item Open-Source, published under the GPL v3 license
		\item $\sim25000$ lines of code
	\end{itemize}
	\end{frame}
	
	\begin{frame}{hhuOS - Features}
		\begin{itemize}
			\item Processes \& Threads
			\begin{itemize}
				\item Round-Robin based preemptive scheduling
				\item Binary files are executed as processes
				\item Each process has its own address space
			\end{itemize}
			\pause
			\item Adress spaces and memory management
			\begin{itemize}
				\item Using x86-Paging mechanism (virtual/physical memory)
				\item Lazy mapping implementation
				\item Kernel address space is always mapped at 3 GiB (not accessible from user space)
				\item Different memory managing algorithms are implemented (Free List, Bitmap, Table)
			\end{itemize}
		\end{itemize}
	\end{frame}

	\begin{frame}{hhuOS - Features}
		\begin{itemize}
			\item Unified Library
			\begin{itemize}
				\item Single codebase for user- and kernel-space library
				\item Functions requiring kernel access are outsourced into an interface (implemented two time - for kernel and user space)
				\item Kernel access from user space via system calls (software interrupts)
			\end{itemize}
			\pause
			\item Further features
			\begin{itemize}
				\item Virtual Filesystem (can mount physical filesystems)
				\item Hardware support (Keyboard, PCI, Floppy AHCI, VESA \& CGA Graphics)
				\item Compatible with UEFI and BIOS systems (BIOS calls are supported)
				\item Own UEFI bootloader (towboot, developed by Niklas Sombert)
				\item Experimental support for networking, thanks to several bachelors thesis (2 drivers and a UDP/IP Stack)
			\end{itemize}	
		\end{itemize}
	\end{frame}